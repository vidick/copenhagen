
\chapter{The Pauli braiding test}

In this lecture we build up our work on nonlocal games and rigidity completed thus far and extend it to design a family of games $(G_n)_{n\geq 1}$ such that near-optimal strategies in $G_n$ require $n$, as opposed to $1$ or $2$, EPR pairs. There are multiple ways in which this can be done. A natural approach would be to consider the parallel repetition of, for example, the Magic Square game. That is, we simply play $n$ copies of the game in parallel and accept if all answers are valid. It can be shown that perfect or even near-optimal success probabilities in this game require strategies that make use of $n$ (or $2n$ in the case of the Magic Square) EPR pairs, see e.g.~\cite{coudron2016parallel}. However, this approach has multiple drawbacks. Firstly, the number of questions and answers in the parallel-repeated game scales exponentially with $n$. Secondly, for all results known the ``robustness'' degrades with $n$, i.e.\ if the success probability is $1-\eps$ then closeness to $n$ EPR pairs is only up to $\poly(n)\cdot \eps^c$ for some small constant $c>0$. Both of these aspects are problematic in applications to complexity. 

In this lecture we give a different method, which is based on leveraging the connection with approximate representation theory we discovered in the previous lecture, and gets around the second issue--there is no dependence of the robustness on $n$. Furthermore, the total number of answers in the game remains a constant. However, the number of questions still scales exponentially. At the end of the lecture we will describe a further improvement that uses only a polynomial number of questions and is based on an ``efficient'' variant of stability. 

Based on our analysis of the CHSH and Magic Square games, to obtain a game that tests for $n$ EPR pairs we should go through the following steps: (i) design a game such that success in the game requires the players to share observables that satisfy all relations that we expect from elements of the group $\mP_n$ generated by $n$ mutually commuting pairs of anti-commuting observables $(X_1,Z_1),\ldots,(X_n,Z_n)$, (ii) apply Theorem~\ref{thm:gh} to obtain closeness of the strategy to an exact representation of this group, (iii) use that, hopefully, all non-trivial representations give us what we want, i.e. something that looks like the tensor product of $n$ copies of the Pauli group, and (iv) apply Lemma~\ref{lem:epr-stable} to conclude that the shared state is locally isometric to the tensor product of $n$ EPR pairs. 

We start in Section~\ref{sec:wh} by studying the group $\mP_n$. Then in Section~\ref{sec:wh-test} we design ``tests'' for the group product relation. 






\section{The Weyl-Heisenberg group}
\label{sec:wh}

Denote by $\mP_n$ the ``$n$-qubit Weyl-Heisenberg group,'' i.e.\ the matrix group generated by $n$-fold tensor products of single-qubit $\sigma_X$ and $\sigma_Z$ matrices. The group $\mP_n$ has cardinality $2\cdot 4^n$, and each element of $\mP_n$ has a unique representative of the form $\pm \sigma_X(a)\sigma_Z(b)$ for $a,b\in\{0,1\}^n$. 
 The irreducible representations of  $\mP_n$ are easily computed from those of $\mP$.  For us the only thing that matters is that the only irreducible representation $g$ which satisfies $g(-\Id)=-g(\Id)$ has dimension $2^n$ and is given by the defining matrix representation. All other irreducible representations have dimension $1$: there are $4^n$ of them, which are all possible products of the four dimension-$1$ irreducible representations of $\mP=\mP_2$.

We now state a version of the Gowers-Hatami theorem tailored to the group $\mP_n$ and a specific choice of presentation for the group relations. 

\begin{corollary}\label{cor:gh}
Let $n,d$ be integer, $\eps \geq 0$, $\ket{\psi} \in \C^d \otimes \C^d$ a permutation-invariant state, $\sigma$ the reduced density of $\ket{\psi}$ on either system,  and $f: \{X,Z\}\times \{0,1\}^n \to Obs(\C^d)$, the set of observables on $\C^d$. For $a,b\in\{0,1\}^n$ let $X(a)=f(X,a)$, $Z(b)=f(Z,b)$, and assume that $X(a)^2=Z(b)^2=I_d$ for all $a,b$. Suppose that the following inequalities hold: consistency
\begin{equation}\label{eq:gh-cons}
 \Es{a} \, \bra{\psi} \big(X(a) \otimes X(a)\big) \ket{\psi} \,\geq\,1-\eps\;,\qquad \Es{b} \,\bra{\psi} \big(Z(b) \otimes Z(b) \big)\ket{\psi}\,\geq\,1-\eps\;,
\end{equation}
linearity
\begin{equation}\label{eq:gh-commute}
 \Es{a,a'} \,\big\|X(a)X(a')-X(a+a')\big\|_\sigma^2 \leq \eps\;,\qquad\Es{b,b'}\, \big\|Z(b)Z(b')-Z(b+b')\big\|_\sigma^2 \leq \eps\;,
\end{equation}
and anti-commutation
\begin{equation}\label{eq:gh-ac}
 \Es{a,b} \,\big\| X(a)Z(b)-(-1)^{a\cdot b} X(a)Z(b)\big\|_\sigma^2\,\leq\,\eps\;.
\end{equation}
Then there exists a $d'\geq d$, an isometry $V:\C^d\to \C^{d'}$, and a representation $g:\mP_n\to U_{d'}(\C)$ such that $g(-I)=-I_{d'}$ and
$$\Es{a,b}\, \big\| X(a)Z(b) - V^*g(\sigma_X(a)\sigma_Z(b))V \big\|_\sigma^2 \,=\, O(\eps)\;.$$
\end{corollary}

Note that the conditions \eqref{eq:gh-commute} and \eqref{eq:gh-ac} in the corollary are very similar to the conditions required of an approximate representation of the group $\mP_n$; in fact it is easy to convince oneself that their exact analogue suffices to imply all the group relations. The reason for choosing those specific relations is that they can be checked using games; see the next subsection for this. Condition \eqref{eq:gh-cons} is necessary to derive the conditions for the application of the Gowers-Hatami theorem from \eqref{eq:gh-commute} and \eqref{eq:gh-ac}, and is also testable; see the proof. 

Nevertheless, note that the number of relations checked in Corollary~\ref{cor:gh} is only $O(4^n)$. This is in contrast to the number of relations that would be checked by applying Theorem~\ref{thm:gh} directly, which is $|\mP_n|^2=O(8^n)$. Thus Corollary~\ref{cor:gh} can be seen as a form of ``efficient stability'' result, where checking a small subset of all relations suffices to obtain a similar conclusion as checking all of them. We return to this topic at the end of the lecture. 


%The conclusion of the corollary implies the conclusion of Theorem~\ref{thm:pbt} simply by using the aforementioned fact on non-trivial representations of $\mP_n$ (see details at the end of Section~\ref{sec:th101}). 

\begin{remark}
Corollary~\ref{cor:gh} can be seen as an extension of the Blum-Luby-Rubinfeld linearity test~\cite{blum1993self}. The latter makes a similar statement, but for the abelian group $\{\pm\sigma_X(a)|\, a\in\{0,1\}^n\} \simeq \Z_2^n$. 
\end{remark}


\begin{proof} 
To apply the Gowers-Hatami theorem we need to construct an $(\eps,\sigma)$-representation $f$ of the group $\mP_n$. Using that any element of $\mP_n$ has a unique representative of the form $\pm \sigma_X(a)\sigma_Z(b)$ for $a,b\in\{0,1\}^n$, we define $f(\pm \sigma_X(a)\sigma_Z(b)) = \pm X(a)Z(b)$. Next we need to verify that $f$ is an approximate representation. Let $x,y\in \mP_n$ be such that $x=\sigma_X(a_x)\sigma_Z(b_x)$ and $y=\sigma_X(a_y)\sigma_Z(b_y)$ for $n$-bit strings $(a_x,b_x)$ and $(a_y,b_y)$ respectively. Up to phase, we can exploit successive cancellations to decompose $(f(x)f(y)^*-f(xy^{-1}))\otimes I$ as
\begin{eqnarray*}
&&\big( X(a_x)Z(b_x)X(a_y)Z(b_y) -(-1)^{a_y\cdot b_x} X(a_x+a_y) Z(b_x+b_y)\big)\otimes I \\ 
&&\qquad =  X(a_x)Z(b_x)X(a_y)\big (Z(b_y)\otimes I - I\otimes Z(b_y)\big)\\
&& \qquad\qquad+ X(a_x)\big(Z(b_x)X(a_y) - (-1)^{a_y\cdot b_x} X(a_y)Z(b_x)\big)\otimes Z(b_y)\\
&& \qquad\qquad+(-1)^{a_y\cdot b_x} \big( X(a_x)X(a_y)\otimes Z(b_y)\big) \big( Z(b_x)\otimes I - I\otimes Z(b_x)\big)\\
&& \qquad\qquad+  (-1)^{a_y\cdot b_x}  \big( X(a_x)X(a_y)\otimes Z(b_y)Z(b_x) - X(a_x+a_y)\otimes Z(b_x+b_y)\big)\\
&& \qquad\qquad+  (-1)^{a_y\cdot b_x} \big(  X(a_x+a_y)\otimes I \big)\big(I\otimes Z(b_x+b_y) - Z(b_x+b_y)\otimes I\big).
\end{eqnarray*}
(It is worth staring at this equality for a little bit. In particular, note the ``player-switching'' that takes place in the 2nd, 4th and 6th lines; this is used as a means to ``commute'' the appropriate unitaries, and is the reason for including \eqref{eq:gh-cons} among the assumptions of the corollary; indeed it is~\eqref{eq:gh-cons} that guarantees that those terms are small.)
Evaluating each term on the state $\ket{\psi}$, taking the squared Euclidean norm, and then the expectation over uniformly random $a_x,a_y,b_x,b_y$, the inequality $\| AB\ket{\psi}\| \leq \|A\|\|B\ket{\psi}\|$ and the assumptions of the theorem let us bound the overlap of each term in the resulting summation by $O({\eps})$. Using $\| (A\otimes I) \ket{\psi}\| = \|A\|_\sigma$ by definition and the triangle inequality we have obtained the bound
$$ \Es{x,y}\,\big\|f(x)f(y)^* - f(xy^{-1})\big\|_\sigma^2 \,=\, O({\eps}).$$
We are now in a position to apply the Gowers-Hatami theorem, which gives an isometry $V$ and exact representation $g$ such that
\begin{equation}\label{eq:gi}
\Es{a,b}\,\Big\|  X(a)Z(b)- \frac{1}{2}V^*\big( g(\sigma_X(a)\sigma_Z(b))  -  g(-\sigma_X(a)\sigma_Z(b))\big)V\Big\|_\sigma^2 \,=\, O({\eps}). 
\end{equation}
Using that $g$ is a representation, $g(-\sigma_X(a)\sigma_Z(b)) = g(-I)g(\sigma_X(a)\sigma_Z(b))$. It follows from \eqref{eq:gi} that $\|g(-I) + I \|_\sigma^2 = O({\eps})$, so we may restrict the range of $V$ to the subspace where $g(-I)=-I$ without introducing much additional error. 
\end{proof}



\section{Testing the Weyl-Heisenberg group relations}
\label{sec:wh-test}



Corollary \ref{cor:gh} makes three assumptions about the observables $X(a)$ and $Z(b)$: that they satisfy approximate consistency \eqref{eq:gh-cons}, linearity \eqref{eq:gh-commute}, and anti-commutation \eqref{eq:gh-ac}. To complete our test we need to show how these relations can be ``certified'' in a two-player game, i.e.\ we want a nonlocal game such that achieving a high enough success probability in the game requires that a certain collection of observables, defined from the players' strategy in the game, satisfies the assumptions of the corollary. There are multiple ways that this can be done; we give one. We start by introducing two stand-alone ``tests.''

\begin{quote}
\textbf{Linearity test:}
\begin{itemize}
\item[(a)] The referee selects $W\in\{X,Z\}$ and $a,a'\in\{0,1\}^n$ uniformly at random. She sends $(W,a,a')$ to one player and $(W,a)$, $(W,a')$, or $(W,a+a')$ to the other, where the sum is taken modulo $2$.\footnote{Elements such as $(W,a)$ are labels sent to the players as their question, and carry no other intrinsic meaning.} 
\item[(b)] The first player replies with two bits $e_1,e_2$, and the second with a single bit $f$. The referee accepts if and only if the player's answers satisfy the natural relation, e.g.\ if the third player received $a+a'$ then it should be that $f=e_1+e_2\mod 2$. 
\end{itemize}
\end{quote}

As always in this section, the test treats both players symmetrically. As a result we can assume that the players' strategy is symmetric, and is specified by a permutation-invariant state $\ket{\psi}\in \C^d \otimes \C^d$ and a measurement for each question: an observable $W(a)$ associated to questions of the form $(W,a)$, and a four-outcome measurement $\{W_{a,a'}\}$ associated with questions of the form $(W,a,a')$. The following exercise asks you to verify this fact. 

\begin{exercise}
A game is called \emph{symmetric} if $\mX=\mY$ and $\mA=\mB$, the distribution on questions $\pi$ is invariant under permutation of the two questions, $\pi(x,y)=\pi(y,x)$ for all $(x,y)$, and the verification predicate is symmetric as well, i.e.\ $V(x,y,a,b)=V(y,x,b,a)$ for all $(x,y,a,b)$. Define a strategy $(\ket{\psi},\{A_{xa}\},\{B_{yb}\})$ to be \emph{symmetric} if $\mH_\reg{A}=\mH_\reg{B}$, $\ket{\psi}$ is invariant under exchange of the two subsystems, and $A_{xa} = B_{xa}$ for all $x,a$. 

Show that whenever a game $\mG$ is symmetric then for any strategy $(\ket{\psi},\{A_{xa}\},\{B_{yb}\})$ that succeeds with some probability $p$ in the game there is a symmetric strategy $(\ket{\tilde{\psi}},\{\tilde{A}_{xa}\})$ that succeeds with the same probability. 
\end{exercise}

The linearity test described above is almost identical to the BLR linearity test, except for the use of the basis label $W\in\{X,Z\}$. The following lemma states conditions that a strategy must satisfy in order to succeed with high probability in the test. 
 
\begin{lemma}\label{lem:com}
Suppose that a family of observables $\{W(a)\}$ for $W\in\{X,Z\}$ and $a\in\{0,1\}^n$, generates outcomes that succeed in the linearity test with probability $1-\eps$, when applied on a symmetric bipartite state $\ket{\psi}\in\C^d\otimes \C^d$ with reduced density matix $\sigma$. Then the following hold: approximate consistency
$$ \Es{a} \, \bra{\psi}\big(X(a) \otimes X(a)\big) \ket{\psi} \,=\,1-O(\eps),\qquad \Es{b} \, \bra{\psi}\big(Z(b) \otimes Z(b) \big)\ket{\psi}\,\geq\,1-O(\eps),$$ 
and linearity 
$$
 \Es{a,a'} \,\big\|X(a)X(a')-X(a+a')\big\|_\sigma^2 = O(\eps),\qquad\Es{b,b'}\, \big\|Z(b)Z(b')-Z(b+b')\big\|_\sigma^2  \,=\, O({\eps}).$$
\end{lemma}

\begin{exercise}
Prove the lemma. (In the case of classical strategies, the conditions are an immediate reformulation of the test. The proof for quantum strategies is not much harder.)
\end{exercise}

Testing anti-commutation can be done using the Magic Square game. 

\begin{quote}
\textbf{Anti-commutation test:}
\begin{itemize}
\item[(a)] The referee selects $a,b\in\{0,1\}^n$ uniformly at random under the condition that $a\cdot b=1$. She plays the Magic Square game with both players, with the following modifications: if the question to the second player is $2$ or $4$ she sends $(X,a)$ or $(Z,b)$ instead; in all other cases he sends the original label of the question in the Magic Square game together with both strings $a$ and $b$. 
\item[(b)] Each player provides answers as in the Magic Square game. The referee accepts if and only if the player's answers would have been accepted in the game. 
\end{itemize}
\end{quote}

Using Theorem~\ref{thm:ms-robust} it is straightforward to show the following. 

\begin{lemma}\label{lem:ac}
Suppose a strategy for the players succeeds in the anti-commutation test with probability at least $1-\eps$, when performed on a symmetric bipartite state $\ket{\psi} \in \C^d \otimes \C^d$ with reduced density matrix $\sigma$. Then the observables $X(a)$ and $Z(b)$ applied by the player upon receipt of questions $(X,a)$ and $(Z,b)$ respectively satisfy 
\begin{equation}\label{eq:ac}
 \Es{a,b:\,a\cdot b=1} \,\big\| X(a)Z(b)-(-1)^{a\cdot b} Z(b)X(a)\big\|_\sigma^2\,=\,O\big(\sqrt{\eps}\big).
\end{equation}
\end{lemma}

\section{Application: an $n$-qubit test}
\label{sec:th101}

We are ready to put all the pieces together and describe a game for testing $n$ EPR pairs. We call this game the ``$n$-qubit Pauli braiding test''. 

\begin{quote}
\textbf{$n$-qubit Pauli braiding test:}
With probability $1/3$ each, 
\begin{itemize}
\item[(a)] Execute the linearity test;
\item[(b)] Execute the anti-commutation test;
\item[(c)] Execute the following consistency test: Send one player a label $W\in\{X,Z\}$ uniformly at random, and to the other $(W,a)$ for $a\in\{0,1\}^n$ chosen uniformly at random. Expect answers $c \in \{0,1\}^n$ and $c'\in\{0,1\}$ respectively. Accept if and only if $a\cdot c = c'$. 
%\item[(c)] Execute a consistency test: select $W\in\{X,Z\}$ and $a\in\{0,1\}^n$ uniformly at random. Send $(W,a)$ to both players and accept if and only if their answers match. 
\end{itemize}
\end{quote}

We sketch why this game indeed tests $n$ qubits. Suppose that a family of observables $W(a)$, for $W\in\{X,Z\}$ and $a\in\{0,1\}^n$, together with projective measurements $\{A_{Xa}\}_{a\in\{0,1\}^n}$ and $\{A_{Zb}\}_{b\in\{0,1\}^n}$ and a state $\ket{\psi}\in\C^d\otimes \C^d$ specify a symmetric strategy that succeeds with probability at least $1-\eps$ in the game. 

Using Lemma \ref{lem:com} and Lemma \ref{lem:ac}, success with probability $1-\eps$ implies that conditions \eqref{eq:gh-cons}, \eqref{eq:gh-commute} and \eqref{eq:gh-ac} in Corollary \ref{cor:gh} are all satisfied, up to error $O(\sqrt{\eps})$. (In fact, Lemma \ref{lem:ac} only implies \eqref{eq:gh-ac} for strings $a,b$ such that $a\cdot b=1$. The condition for string such that $a\cdot b=0$ follows from the other conditions.) The conclusion of the corollary is that there exists an isometry $V$ such that the observables $X(a)$ and $Z(b)$ satisfy 
$$\Es{a,b}\, \big\| X(a)Z(b) - V^*g(\sigma_X(a)\sigma_Z(b))V \big\|_\sigma^2 \,=\, O(\sqrt{\eps})\;,$$
for some non-trivial representation $g$ of $\mP_n$. Using what we know of non-trivial representations of this group, it follows that there is an isometry which approximately maps $X(a)$ and $(b)$ to $\sigma_X(a)\otimes \Id$ and $\sigma_Z(b) \otimes \Id$ respectively. Finally, we obtain the $n$ EPR pairs using the consistency relations from part (c) of the test and Lemma~\ref{lem:epr-stable}. 



\section{Efficient stability}

We start with the following definition, that refines Definition~\ref{def:approx-rep} to the case where an approximate representation is only defined to approximately respect the group multiplication for a pre-specified set of defining relations. 

\begin{definition}\label{def:approx-rep-2}
Given a finite presentation $G = \langle x_1,\ldots x_m | R_1,\ldots, R_t \rangle$, an integer $d\geq 1$, $\eps\geq 0$, and a $d$-dimensional positive semidefinite matrix $\sigma$ with trace $1$, an $(\eps,\sigma)$-representation of $G$ is a collection $X_1,\ldots,X_m\in  U_d(\C)$ such that 
\begin{equation}\label{eq:gh-condition-b}
\frac{1}{t}\sum_{j=1}^t \big\| R_j(\{X_i\}) - \Id \big\|_\sigma^2 \,\leq\,2\;\eps\;,\footnote{The factor $2$ on the right-hand side is unimportant; we include it for consistency with~\eqref{eq:gh-condition-norm}.}
\end{equation} 
where $R_j(\{X_i\})$ denotes the relation $R_j$ with each occurrence of a generator $x_i$ replaced by $X_i$.
\end{definition}

In the case where the presentation $G = \langle x_1,\ldots x_m | R_1,\ldots, R_t \rangle$ is the ``exhaustive'' presentation, with every element of $G$ included in the generators and all relations of the form $x\cdot y\cdot (xy)^{-1}=1$, then we recover Definition~\ref{def:approx-rep}. If the representation is ``near-exhaustive'', e.g. every group element (resp. relation of the form $x\cdot y\cdot (xy)^{-1}=1$) can be obtained by multiplying at most $k$ generators (resp. chaining $k$ defining relations), then it is not hard to see that an $(\eps,\sigma)$ approximate representation in the sense of Definition~\ref{def:approx-rep-2} implies an $(\eps',\sigma)$ approximate representation in the sense of Definition~\ref{def:approx-rep} for some $\eps' = \poly(k)\eps$. What is perhaps much more surprising is that in some cases we can find very ``efficient'' presentations such that the dependence on $k$ is far milder. The following is shown in~\cite{ji2020quantum}.

\begin{theorem}
There is a presentation $\mP_n = \langle X_1,\ldots,X_m | R_1,\ldots,R_t \rangle$ such that $m,t=\poly(n)$ and any $(\eps,\sigma)$-representation in the sense of Definition~\ref{def:approx-rep} is an $(\eps',\sigma)$-representation in the sense of Definition~\ref{def:approx-rep}, for some $\eps' = \poly\log(n)\cdot \poly(\eps)$. 
\end{theorem}

Note that since $|\mP_n|=O(4^n)$ but $m=\poly(n)$, there are elements of $|\mP_n|$ that cannot be written as a product of less than $\poly(n)$ generators $X_1,\ldots,X_m$. Na\"ively extending a representation in the sense of Definition~\ref{def:approx-rep} to a map defined over the full group would lead to an $(\eps',\sigma)$-representation in the sense of Definition~\ref{def:approx-rep} where $\eps' = \poly(n)\cdot \eps$. The theorem provides an exponential improvement over this naive method. 

Based on this theorem it is possible to define a nonlocal game that provides essentially the same guarantees as the Pauli braiding test, but now the total number of questions in the game is polynomial in $n$, the number of EPR pairs being tested. (See~\cite[Section 7]{ji2021mip} for a complete description of this game.) This exponential improvement turns out to be crucial for the applications to complexity which we discuss in the next lecture. 
